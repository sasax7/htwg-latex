\chapter*{Abstract}
\setheader{Abstract}

Buildings account for approximately 30\% of global final energy consumption, while empirical studies estimate that between 4\% and 18\% of building energy use is attributable to anomalies such as technical faults, control and scheduling errors, and behaviour-induced energy misuse, which result in avoidable operational inefficiencies.

This thesis presents an integrated methodology for contextual anomaly detection in multivariate, non-stationary building-energy time series, enabling financial-impact estimation and automated root-cause attribution. The approach is fully implemented within an existing IoT building-management platform and includes a production-ready frontend for anomaly visualization and operational analysis.

To address structural limitations of existing anomaly-detection benchmarks for building-energy data, a dedicated evaluation dataset was constructed using the BOPTEST simulation environment, comprising a clean baseline and systematically injected multivariate, context-dependent anomaly scenarios. Multiple detection methods, including statistical, deep-learning, and foundation-model-based approaches, were evaluated on this benchmark.

The results indicate that stochastic prediction models with probabilistic output distributions are more suitable than deterministic point predictors for modelling multimodal building-energy behaviour and identifying contextual anomalies. Chronos-2 enables the practical application of time-series foundation models to multivariate energy telemetry without per-asset training, while mixture-density modelling was identified as a promising architectural direction for future research. The findings establish a methodological basis for a universal energy foundation model supporting zero-shot anomaly detection and standardized baseline comparison in accordance with IPMVP.
