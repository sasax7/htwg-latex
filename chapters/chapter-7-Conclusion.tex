\chapter{Conclusion}
\label{chap:conclusion}

The research conducted in this thesis successfully established a robust methodology and technical framework for the automated detection and quantification of energy anomalies in building environments. By addressing the statistical complexities of building telemetry—specifically its non-stationary and multi-modal nature—the project provided a solution that transcends the limitations of traditional deterministic forecasting.

The investigation into predictive modelling paradigms revealed that standard sequential forecasting is susceptible to error propagation and the adaptation paradox. It was shown that autoregressive models often incorporate anomalous data into their internal state, which leads to signal instability and the masking of sustained faults. To resolve these issues, a stochastic approach was developed that utilizes the \textbf{Chronos-2} foundation model within a feature-driven inference strategy. This methodology establishes a probabilistic normative operational band, allowing for the reliable identification of \textbf{Multivariate Context Point Anomalies (MCPA)}\textemdash deviations that are only identifiable through the joint analysis of consumption and exogenous drivers such as weather and occupancy.

The technical realization was achieved through a distributed, polyglot architecture. A high-performance \textbf{Scala} microservice managed multi-tenant isolation and data orchestration, while a \textbf{Python} endpoint hosted on \textbf{Azure Machine Learning} provided the required predictive capacity. The integration of a hierarchical \textbf{Root Cause Analysis (RCA)} and \textbf{Generative AI} synthesis transformed raw detection results into actionable operational intelligence. It was demonstrated that by attributing financial impact to specific assets and generating natural-language remediation steps, the system provides facility managers with a transparent tool for energy waste mitigation.

The implementation of specialized frontend modules—including the anomalies list, status management, and temporal heatmaps—ensures that detection events are integrated into standard maintenance workflows. The capacity for human-in-the-loop feedback, where users validate anomalies as confirmed or false, establishes a mechanism for continuous data cleansing and model refinement. This ensures that the system remains accurate as building characteristics evolve over time.

Ultimately, this research moved the \textbf{Eliona} platform from a state of reactive alarm management to proactive, intelligence-driven energy monitoring. The shift from point-based detection to stochastic distribution modeling enables a precise calculation of financial residuals, providing organizations with quantifiable data to support their sustainability and decarbonization objectives. While future work remains regarding the inclusion of control-layer states and the development of in-context forecasters, this project provides a scientifically validated foundation for the next generation of energy management systems in the building-automation sector.