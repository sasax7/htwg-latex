\chapter{Introduction}
\label{chap:introduction}

\section{Motivation and Problem Statement}
Energy efficiency and resource conservation are paramount challenges in the context of global climate change and rising operational costs. In both industrial and residential settings, energy consumption data is being collected at an unprecedented scale, largely due to the widespread adoption of smart meters and Internet of Things (IoT) devices \cite{SomeSmartMeterPaper2023}.

While this data holds immense value, it also presents a significant challenge: identifying consumption patterns that deviate from the norm. These \emph{anomalies} can represent critical information, such as:
\begin{itemize}
    \item Equipment malfunction or failure,
    \item Inefficient operational processes,
    \item Data integrity issues from faulty sensors, or
    \item Potential for energy savings and process optimization.
\end{itemize}

The manual analysis of these vast, high-velocity time series datasets is impractical. Automated methods are required to detect these anomalies reliably and in near real-time. However, defining a "normal" energy profile is complex due to factors like seasonality (daily, weekly, yearly cycles), weather dependencies, and stochastic user behavior. This complexity makes simple threshold-based detection methods ineffective, leading to a high rate of false positives or missed detections.

\section{Research Goal and Objectives}
The primary goal of this thesis is to design, implement, and evaluate a robust system for anomaly detection in energy consumption time series data. This work aims to compare different algorithmic approaches to identify the most suitable method for a given energy dataset.

To achieve this goal, the following key objectives are defined:
\begin{enumerate}
    \item \textbf{Literature Review:} To investigate the fundamentals of time series anomaly detection and review the current state-of-the-art (Related Work) specific to energy data.
    \item \textbf{Data Preprocessing:} To select a suitable energy dataset and develop a preprocessing pipeline to handle missing values, normalize data, and engineer relevant features.
    \item \textbf{Model Implementation:} To implement and train several detection models, likely including a statistical baseline (e.g., ARIMA) and one or more machine learning models (e.g., Isolation Forest, Autoencoder).
    \item \textbf{Evaluation:} To define and apply appropriate evaluation metrics (e.g., Precision, Recall, F1-Score) to systematically compare the performance of the implemented models.
\end{enumerate}

\section{Scope and Delimitations}
This thesis focuses on \textbf{unsupervised anomaly detection}, as labeled anomaly data is rare and expensive to obtain in real-world energy scenarios. The primary input will be univariate energy consumption time series. While external factors like weather or building occupancy are acknowledged as influential, their integration as exogenous variables is considered outside the primary scope of this work but will be discussed as potential future work.

Furthermore, this work is concerned with the \textbf{detection} of anomalies, not their \textbf{diagnosis}. The system will flag a data point or sequence as anomalous, but it will not perform a root-cause analysis of the anomaly's origin.

\section{Structure of the Thesis}
This document is organized into six chapters:

\begin{description}
    \item[Chapter 1 (Introduction)] motivates the research topic, defines the core problem, and introduces the primary objectives regarding multivariate context point anomalies (MCPA). It further outlines the functional and non-functional requirements for the anomaly detection system.
    \item[Chapter 2 (Foundations)] establishes the theoretical background for this work. It characterizes building energy telemetry as a multivariate, multi-mode time series, analyzes the causal chain of energy consumption, and examines temporal and statistical properties such as autocorrelation, seasonality, and mixture distributions. The chapter then formalizes anomaly types and learning paradigms, introduces a taxonomy of detection methods with an emphasis on prediction-based approaches, and defines benchmarking concepts including ground truth, confusion matrices, and evaluation metrics.
    \item[Chapter 3 (Related Work)] reviews and structures the state of the art in time series anomaly detection with a focus on building-related use cases. It discusses the reliability of existing benchmarks, highlighting systemic data flaws and metric biases and motivating the adoption of VUS-PR. The chapter then compares statistical, deep learning, and foundation-model-based methods (including CNN variants, OmniAnomaly, and time-series foundation models such as Chronos-2), and concludes by identifying open research gaps and synthesizing the objectives addressed in this thesis.
    \item[Chapter 4 (The Benchmark: Methodology and Selection)] presents the design of the custom benchmark tailored to building energy telemetry. It begins with an initial proof of concept to motivate the chosen experimental setup, then details the generation of synthetic datasets with varying baseline lengths (e.g., two weeks, three months, and one year), the labeling strategy for MCPA and global outliers, and the experimental comparison of candidate models such as CNN-based architectures, OmniAnomaly, and Chronos-2. The chapter reports results using the VUS-PR metric and derives the final model choice based on overall performance and seasonal generalization.
    \item[Chapter 5 (Technical Realization and System Integration)] describes the deployment of the selected model as a high-performance service and its integration into a microservice-based architecture. It explains the Scala-based orchestration of data preprocessing and anomaly scoring, the bidirectional coupling with the IoT platform, and the persistence strategy for raw telemetry and detected events. The chapter also introduces the root cause analysis logic for attributing anomalies to specific sensors or subsystems and outlines the frontend implementation for monitoring building energy health.
    \item[Chapter 6 (Conclusion and Future Work)] summarizes the core findings of the research, evaluates the effectiveness of the benchmark and deployed system, and reflects on current limitations. It then sketches a conceptual design for a next-generation multivariate time-series foundation model to address remaining challenges such as seasonal drift and scalability.
\end{description}