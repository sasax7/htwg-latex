\chapter{State of the Art (Related Work)}
\label{chap:related_work}

% This chapter reviews existing literature to situate this thesis within the
% current research landscape. It reviews specific algorithms (Section 3.1),
% their application to the energy domain (Section 3.2), and concludes
% by identifying the research gap this thesis will address (Section 3.3).

\section{Implementations of Detection Algorithms}
\label{sec:sota_algorithms}
% This section reviews specific, named algorithms from the literature,
% building on the categories defined in Section 2.4.

\subsection{Statistical and Classical Machine Learning Methods}
\label{sec:sota_classical}
% TODO: Describe model-based statistical approaches (from your old template)
These methods are often used as efficient baselines.
\begin{itemize}
    \item \textbf{ARIMA/SARIMA:} (Autoregressive Integrated Moving Average). Models the time series and flags points with high residuals as anomalous \cite{ExampleARIMAPaper}.
    \item \textbf{Isolation Forest (IF):} An ensemble method that "isolates" outliers. It is very efficient and scales well \cite{Liu2008_IF}.
    \item \textbf{One-Class SVM (OC-SVM):} A kernel-based method that learns a boundary around the "normal" data \cite{Scholkopf2001_OCSVM}.
    \item \textbf{DBSCAN:} A density-based clustering algorithm that identifies points in low-density regions as outliers \cite{Ester1996_DBSCAN}.
\end{itemize}

\subsection{Deep Learning Methods}
\label{sec:sota_deep_learning}
% TODO: Describe neural network approaches (from your old template)
Deep learning is powerful for capturing complex, non-linear patterns.
\begin{itemize}
    \item \textbf{LSTM-based Autoencoders (LSTM-AE):} This architecture combines the sequential power of LSTMs with the reconstruction-based anomaly detection of an Autoencoder. It is a very common approach for time series \cite{Malhotra2016_LSTM_AD}.
    \item \textbf{LSTM-based Prediction:} Simple LSTM networks are trained to predict the next time step; a large prediction error indicates an anomaly.
\end{itemize}

\subsection{Foundation Models for Time Series}
\label{sec:sota_foundation_models}
% TODO: This is the newest category.
A recent paradigm shift involves using large, pre-trained Foundation Models.
\begin{itemize}
    \item \textbf{Probabilistic Forecasting Models:} Models like \textbf{Moirai} \cite{Moirai2024} and \textbf{Chronos} \cite{Chronos2024} are pre-trained on vast public datasets (including energy data). They can provide probabilistic (quantile) forecasts in a zero-shot or few-shot (finetuned) manner, making them ideal for prediction-based anomaly detection.
    \item \textbf{General-Purpose Models:} \textbf{MOMENT} \cite{Goswami2024_MOMENT} is a foundation model pre-trained for multiple tasks, including explicit anomaly detection.
\end{itemize}

\section{Applications in Energy Anomaly Detection}
\label{sec:sota_applications}
% TODO: This is the most important part of this chapter.
% You must find, read, and summarize *actual* papers.
% This section reviews existing research that specifically applies anomaly
% detection methods (from 3.1) to energy data (from 2.1).

\subsection{Applications in Commercial and Residential Buildings}
\label{sec:sota_buildings}
% TODO: Find 3-5 papers that do this.
For example: \cite{ExamplePaperBuilding2021} used an LSTM-based Autoencoder to detect anomalies in HVAC systems of a university building. They found...
Another study by \cite{ExamplePaperSmartHome2022} compared Isolation Forest and OC-SVM for smart home data and found that...
Lorem ipsum dolor sit amet, consectetur adipisicing elit, sed do eiusmod tempor incididunt ut labore et dolore magna aliqua.

\subsection{Applications in Industrial and Grid-Level Data}
\label{sec:sota_grid}
% TODO: Find 3-5 papers that do this.
For example: \cite{ExamplePaperGrid2020} focused on anomaly detection in smart grid phase measurements (PMU data) using...
Lorem ipsum dolor sit amet, consectetur adipisicing elit, sed do eiusmod tempor incididunt ut labore et dolore magna aliqua.

\subsection{Comparative Studies and Benchmarks}
\label{sec:sota_benchmarks}
% TODO: Find 1-2 papers that *compare* many methods.
A recent benchmark by \cite{ExampleBenchmarkPaper2023} evaluated five different algorithms on a public energy dataset. Their findings indicate that... The TSB-UAD benchmark \cite{Paparrizos2022_TSB} also provides extensive comparison, although...

\section{Research Gap and Contribution}
\label{sec:research_gap}
% TODO: Based on the related work (3.1, 3.2), what is missing?
% This is the justification for *your* thesis.
%
% Example gaps:
% - "While many studies use classical DL (like LSTMs), the application of *Foundation Models* (like Moirai) for anomaly detection in a real-world commercial building context is still unexplored."
% - "Most research focuses on public benchmarks. There is a gap in applying these methods to a specific, proprietary IoT platform and integrating them into a production environment."
% - "There is no direct comparison between a finetuned Foundation Model (Moirai) and a classical, efficient model (Isolation Forest) for the specific task of HVAC anomaly detection."
%
Lorem ipsum dolor sit amet, consectetur adipisicing elit, sed do eiusmod tempor incididunt ut labore et dolore magna aliqua.

% TODO: State your contribution.
This thesis aims to fill this gap by...
(e.g., "...by designing, implementing, and evaluating an anomaly detection system integrated into the [Firma X] IoT platform. It will systematically compare a state-of-the-art Foundation Model (Moirai) using probabilistic forecasting against a highly efficient classical model (Isolation Forest)...")
This leads directly to the methodological choices in the next chapter.
